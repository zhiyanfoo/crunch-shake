\documentclass[12pt]{article}
\usepackage{hyperref}
\usepackage{tabularx}
\usepackage{booktabs}
\usepackage{caption}
\hypersetup{
    colorlinks=true,
    linkcolor=blue,
    filecolor=magenta,      
    urlcolor=blue,
    citecolor=black,
}

\urlstyle{same}
\usepackage{csquotes}
\usepackage{graphicx}


\begin{document}

\title{Does Shakespeare Pass the Bechdel Test?}
\date{November 11, 2016}
\author{Zhiyan Foo}
\maketitle
\begin{abstract}
The Bechdel Test is measure of gender bias in a piece of artistic work. This
    paper presents the findings when the Bechdel Test is applied to
    Shakespeare's plays computationally. The results of the experiment show
    that Shakespeare's plays are overwhelmingly focused on the depiction of the
lives of its male characters. \end{abstract}
\section{Background and Significance}
\label{sec:background_and_significance}

Despite the recent success of films such as \emph{The Hunger Games} and
\emph{Star Wars: The Force Awakens}, most of our cultural output is
disproportionally focused on the male lives and their stories.\cite{atlantic}. One popular
measure of this gender bias is the Bechdel Test\cite{polygraph}. The test
specifies 3 criteria for a piece to pass. 
\begin{enumerate}  
\item It must contain two named female characters
\item They must have a conversation
\item The conversation must not be about a man
\end{enumerate}
While this is not a particularly high bar, a surprising amount of films---the
original medium the test was applied to\footnote{see
\url{http://bechdeltest.com/} for a crowdsourced effort to classify movies by
the Bechdel Test.}---fail it. Gender disparities however are not a recent
phenomena and this article looks at whether Shakespeare's plays in particular
pass the Bechdel Test. This would not be done manually however but as a
showcase for a new software package, \textbf{crunch-shake}\footnote{\url{https://github.com/zhiyanfoo/crunch-shake}}.

\section{Methodology}
\label{sec:methodology}

\subsection{Modified Criteria}
\label{sub:crunch_shake_s_new_criteria}
The criteria used by crunch-shake is a modified version of the original
Bechdel Test. This is to both accommodate and utilize the computational nature
of crunch-shake. 
\begin{center}
    \begin{tabularx}{\textwidth}{ X X X }
    \toprule
    Original Criteria & crunch-shake Criteria & Reason for Change \\ \midrule
        Contain two named female characters & Contain two female characters
        that are in the upper 50\% of notable characters. See
        \hyperref[sub:gender]{Gender} and
        \hyperref[sub:notability]{Notability}.
        & The requirement that the two female characters be named is just a
        proxy for whether the character is significant to the piece, and using
        social network, graphing algorithms and the number of lines the
        character has, significance of the character can be determined directly.
        \\ \hline
    The two named female characters must have a conversation & In any scene,
        two notable female characters must speak in the presence of one
        another.
        See \hyperref[sub:presence]{Presence}.
        & It is hard to algorithmically determine when two characters are in a
        'conversation' with one another. So the two females might be talking to
        a male, not each other, but unfortunately the scene will still
        'pass'. \\ \hline
    The conversation must not be about a man & In their 'conversation', the two
        notable females must not utter a word related to romantic relationships
        or mention a male. See \hyperref[sub:blacklist]{Blacklist}. & While
        there's a lot of subtext that an algorithm can miss out, a blacklist of
        words takes care of the more obvious
        cases. \\
    \bottomrule
    \end{tabularx}
\end{center}
Whereas traditionally, the Bechdel Test
has only been applied to females, the computational nature of the test means
that the test can be also easily be applied to males, and the results of the
two tests compared. So for example the male version of the Bechdel Test would
check to see if there exists at least two notable males who talk about
something other than women in a scene. 

\subsection{Gender}
\label{sub:gender}

The gender of each character is determined manually.\footnote{\url{https://github.com/zhiyanfoo/crunch-shake/tree/master/crunch-shake/gender}}
While it is easy enough to determine the gender of the named characters, the
unnamed characters proved to be a bit harder. Some gender assignments were
easy, such as `groom' or `maid'. Others such as `Soldier', while could
represent both genders in today's society, where for the most part male in
Shakespeare's time and so classified as such. The more ambiguous designations
such as 'Citizen' while were for the most part were probably intended to be male,
are left as undetermined, and for the purposes of the algorithm, these people
might as well not exist.

\subsection{Notability}
\label{sub:notability}

Whether a character is notable or not is dependent on how each character
scores on 4 metrics: lines by character, out degree, page rank, betweenness.
The last 3 metrics are network algorithms. crunch-shake uses the
implementation found in the python package networkx
\footnote{\url{https://networkx.readthedocs.io/en/stable/reference/algorithms.centrality.html}}.
\begin{center}
    \begin{table}
    \caption*{Table 1: Metrics used by crunch-shake to evaluate
    notability}
    \centering
    \begin{tabular}{ l l p{5cm} }
        \toprule
    Metric & Weight & Algorithm \\ \midrule
        lines by character & 62.5\% & Take the number of lines a character has
        and divide it by the lines of speech the character with
        the most lines has.\\ \hline
        out degree & 12.5\% & networkx's implementation \\ \hline
        page rank & 12.5\% & networkx's implementation \\ \hline
        betweenness & 12.5\% & networkx's implementation \\
    \bottomrule
    \end{tabular}
    \end{table}
\end{center}
In this case, the network vertices here are speaking characters, with directed
edges representing when a character speaks to another character, with more
lines of speech indicating a stronger connection.

The use of network algorithms to classify importance of characters was taken
from a paper on Game Of Thrones(GOT)\cite{got}, a TV series. Since
Shakespeare's plays feature far fewer characters than in GOT and since those
characters are far more interconnected than in GOT, this methodology does not
work as well for this paper, which is why the network algorithms are assigned
such low weights, as shown in Table 1.


\subsection{Presence}
\label{sub:presence}

\subsubsection{Methods}
\label{ssub:method}


There are two ways the algorithm in crunch-shake knows that a character
is in a scene. The first is through stage directions. For example, if the
algorithm sees the stage direction,
\begin{displayquote}
Enter CAPULET in his gown, and LADY CAPULET.\footnote{From Act
    I, Scene I in \emph{Romeo and Juliet}}
\end{displayquote}
it would note that CAPULET and LADY CAPULET have entered. Similarly, the
algorithm will remove the character from the scene if it sees \textquote{Romeo
Exit}

In the second method, if the algorithm sees that Romeo speaks a line, even if he never
`entered' the scene---so that scene started \textit{in media res}---it would
take it that Romeo was always there. Romeo would be assumed to be present until
it it sees \textquote{Romeo Exit} or the scene ends.

\subsubsection{Limitations}
\label{ssub:limitations}

There are limitations to the algorithm however. For example if it sees,
\begin{displayquote}
Exeunt all but MONTAGUE, LADY MONTAGUE, and BENVOLIO.\footnote{From
the same scene}
\end{displayquote}
it would erroneously note that MONTAGUE, LADY MONTAGUE, and BENVOLIO have all
exited as it sees 'exeunt' while keeping the rest of the characters as it is
not smart enough to interpret 'but all'. 

Also if a character enters, exit and re-enters, the algorithm will note only
the first entrance and the last exit. Finally sometimes the play directions do
not refer to the characters by name. Take this example from Act IV, Scene III
of \emph{The Taming of the Shrew}, where the SERVANT is not mentioned directly,
\begin{displayquote}
Enter four or five Serving-men
\end{displayquote}
yet is suppose to enter the scene. These errors are unlikely to skew the data
in favor of males or females however.

\subsection{Blacklist}
\label{sub:blacklist}

The list of forbidden words is tabulated below. 
\begin{description}
    \item[romantic relationships] marriage,
        matrimony, courting, love, wedlock, sex, sexual, intercourse
    \item[partners] partner, spouse, lover,
        admirer, fiancé, amour, inamorato, betrothed. 
    \item[specific to females] boyfriend, husband
    \item[specific to males] girlfriend, wife
\end{description}

In addition the names of all the male or female characters would also be
blacklisted. The idea is not that it is 'wrong' for script writers to ever have
their characters mention the opposite sex, but rather not ever scene involving
two people from the same gender should have them discussing the opposite
gender.

\subsection{Precaution}
\label{sub:Precautions}

In order not to bias the data by, for example changing the blacklisted words,
or changing the weights given to each metrics, I wrote the specifications
before running crunch-shake on Shakespeare's plays, with the exception of
\emph{All's Well that Ends Well} and \emph{Romeo and Juliet}, which I needed
for debugging and checking feasibility purposes. As a result I will be omitting
these two plays from the results. 

\section{Results}
\label{sec:results}

\begin{center}
\begin{table}
    \caption*{Table 2: Ranking of Shakespeare's Play's by FB (F) \%}
\resizebox{\textwidth}{!}{%
    \begin{tabular}{  m{0.8cm}  m{5cm} m{1.3cm}  m{1.3cm} m{1.3cm}
        m{1.3cm} m{1.3cm} m{1.3cm} }
    \toprule
        & & FB & FB & NC & NC & BL & BL\\ 
        Rank  &  Play& (F) \% &  (M) \% & (F) \% &  (M) \% & (F) \% &  (M) \% \\ \midrule
1 & Comedy of Errors & 90.91 & 81.82 & 45.45 & 18.18 & 45.45 & 63.64 \\ 
2 & King Lear & 92.31 & 61.54 & 80.77 & 15.38 & 11.54 & 46.15 \\ 
3 & Winter's Tale & 92.86 & 92.86 & 71.43 & 21.43 & 21.43 & 71.43 \\ 
4 & Henry IV, part 2 & 94.74 & 47.37 & 89.47 & 15.79 & 5.26 & 31.58 \\ 
5 & As You Like It & 95.45 & 81.82 & 54.55 & 36.36 & 40.91 & 45.45 \\ 
6 & Henry V & 95.65 & 43.48 & 91.30 & 13.04 & 4.35 & 30.43 \\ 
7 & Cymbeline & 96.15 & 80.77 & 88.46 & 30.77 & 7.69 & 50.00 \\ 
8 & Antony and Cleopatra & 97.62 & 57.14 & 76.19 & 23.81 & 21.43 & 33.33 \\ 
9 & Much Ado About Nothing & 100.0 & 88.24 & 64.71 & 23.53 & 35.29 & 64.71 \\ 
10 & Macbeth & 100.0 & 67.86 & 78.57 & 39.29 & 21.43 & 28.57 \\ 
11 & Timon of Athens & 100.0 & 41.18 & 100.0 & 23.53 & 0.00 & 17.65 \\ 
12 & Twelfth Night & 100.0 & 72.22 & 77.78 & 38.89 & 22.22 & 33.33 \\ 
13 & Henry IV, part 1 & 100.0 & 47.37 & 100.0 & 15.79 & 0.00 & 31.58 \\ 
14 & Richard III & 100.0 & 56.00 & 76.00 & 28.00 & 24.00 & 28.00 \\ 
15 & Othello & 100.0 & 86.67 & 60.00 & 6.67 & 40.00 & 80.00 \\ 
16 & Henry VI, part 3 & 100.0 & 39.29 & 100.0 & 7.14 & 0.00 & 32.14 \\ 
17 & Coriolanus & 100.0 & 65.52 & 82.76 & 24.14 & 17.24 & 41.38 \\ 
18 & Midsummer Night's Dream & 100.0 & 77.78 & 44.44 & 0.00 & 55.56 & 77.78 \\ 
19 & The Tempest & 100.0 & 66.67 & 66.67 & 0.00 & 33.33 & 66.67 \\ 
20 & Two Gentlemen of Verona & 100.0 & 95.00 & 85.00 & 45.00 & 15.00 & 50.00 \\ 
21 & Merry Wives of Windsor & 100.0 & 91.30 & 65.22 & 21.74 & 34.78 & 69.57 \\ 
22 & Measure for Measure & 100.0 & 88.24 & 70.59 & 35.29 & 29.41 & 52.94 \\ 
23 & Hamlet & 100.0 & 70.00 & 85.00 & 15.00 & 15.00 & 55.00 \\ 
24 & Henry VI, part 1 & 100.0 & 40.74 & 96.30 & 11.11 & 3.70 & 29.63 \\ 
25 & Julius Caesar & 100.0 & 44.44 & 100.0 & 11.11 & 0.00 & 33.33 \\ 
26 & Troiles and Cressida & 100.0 & 66.67 & 100.0 & 8.33 & 0.00 & 58.33 \\ 
27 & Richard II & 100.0 & 57.89 & 100.0 & 21.05 & 0.00 & 36.84 \\ 
28 & Henry VI, part 2 & 100.0 & 41.67 & 100.0 & 4.17 & 0.00 & 37.50 \\ 
29 & Love's Labour's Lost & 100.0 & 100.0 & 66.67 & 0.00 & 33.33 & 100.0 \\ 
30 & Taming of the Shrew & 100.0 & 92.86 & 64.29 & 0.00 & 35.71 & 92.86 \\ 
31 & Henry VIII & 100.0 & 88.24 & 94.12 & 17.65 & 5.88 & 70.59 \\ 
32 & Merchant of Venice & 100.0 & 78.95 & 57.89 & 36.84 & 42.11 & 42.11 \\ 
33 & Titus Andronicus & 100.0 & 71.43 & 92.86 & 0.00 & 7.14 & 71.43 \\ 
34 & King John & 100.0 & 37.50 & 87.50 & 6.25 & 12.50 & 31.25 \\ 
35 & Pericles & 100.0 & 84.21 & 73.68 & 15.79 & 26.32 & 68.42 \\ 
    \bottomrule
        \\
    \end{tabular}}
    \caption*{
    Legend  
    \begin{tabular}{|l l}
      FB & \% Scenes that fail the Bechdel Test \\
      NC & \% Scenes with 'No Conversations' \\
      BL & \% Scenes with 'Blacklisted' conversation \\
    \end{tabular}
    }
\end{table}
\end{center}

\begin{table}[htpb]
    \centering
    \caption*{Table 3: Results Across All Plays}
    \label{tab:3}
    \begin{tabular}{c c c c c c}
        & PFB & SFB \% & SNC \% &  SBL \% & COG \% \\
        Female & 27 & 98 & 80 &17 & 91 \\
        Male & 1 & 67 &58 & 20 & 58 \\
        \\

    \end{tabular}

    \caption*{
    Legend  
    \begin{tabular}{|l l}
      PFB & Plays that fail the Bechdel Test \\
      SFB & Scenes that fail the Bechdel Test \\
        SNC & Scenes with `No Conversations'  \\
        SBL & Scenes with `Blacklisted' conversation \\
        COG & Conversation mentions the other gender \\
    \end{tabular}
    }

\end{table}
\subsection{Evaluation of Results}
\label{sub:evaluation_of_data}


Even though I wasn't expecting Shakespeare to give equal weight to both sexes
in his plays, the extreme discrepancies that crunch-shake found still surprised
me \footnote{For the figures in Table 3 (SFB, SNC, SBL and COG) these are
calculated by considering all scenes in Shakespeare's plays again, not by
averaging the results in Table 2. This is so as to not diminish the impact of
longer works while exaggerating the effect of shorter works.} For females, 27
out of 35 of his plays do not pass the Bechdel Test \footnote{For a play to
fail, all scenes must fail the Bechdel Test.}, whereas for males, only one of
the plays, \emph{Love's Labour's Lost}, fails the Bechdel Test.  And even that
was only because the men in it managed to talk about women in every scene.
However it needs to be acknowledged the failure rate of the Bechdel Test for
males per scene---as opposed to per play---was quite high. In particular, 67\%
of the scenes did not pass the Bechdel Test for males\footnote{If a scene
failed, either the scene had `no conversation', or the conversation included
`blacklisted' words}.  Not nearly as bad as the figure for females, 98\% , but
not a small figure either. However if you take a closer look at the make up of
these figures, it is evident that the scenes failed for females and males for
different reasons.  In 47\% of the scenes, males failed because they talked
about women\footnote{crunch-shake considers a scene to have failed if if two
members of the same gender have a conversation which mentions the opposite
gender. It does this by having a blacklist of words the conversation must not
include.}, whereas the figure is 17\% for women. This is not because women tend
not to talk about men, in fact in 91\% of the scenes in which they talk, they
mention men. This figure is 58\% for men.  However in general women don't get
to talk to each other very often. In 80\% of the scenes, there was no
conversation between two notable females \footnote{crunch-shake determines a
conversation to have occurred if two notable characters from the same gender
both speak within a scene.}, whereas this occurred on 20\% of the time for
males. 

\subsection{Examination of Scenes}
\label{sub:examination_of_scenes}

Taking a closer look at the small number scenes which the passed for females,
one realizes it might be to soon to even celebrate even those small
achievements.  Randomly choosing 5 out of the 8 scenes that passed, 2 of them
(\emph{Comedy of Errors} and \emph{As You Like It}) do contain two women
speaking, but it is about a man, they just don't mention him by name. In
\emph{Henry V}, Katherine and her lady-in-waiting, Alice, do speak, but it is
in French, which the algorithm does not understand. In \emph{Winter's Tale},
two women do speak, just not to each other but rather to the King of Sicily.
Finally in \emph{Henry IV, part 2}, a prostitute, Doll Tearsheet, and a brothel
owner, Mistress Quickly, are not having a conversation as so much being dragged
around by a beadle. As such none of these scenes can be said to have passed the
Bechdel Test.

Looking at the scenes that passed for males, the first from \emph{Henry VIII},
involves two noblemen making fun of the newest fashion from France. The second
from \emph{Henry IV, part 2}, involves Cade and the King's ambassadors
alternatively giving speeches to try get the mob to come to their respective
sides. The third scene from \emph{Othello} involves the eponymous protagonist
giving instructions to his subordinates, as he is about to scout the camp's
fortification. The fourth scene, from \emph{Pericles}, revolves around Thaliard
eavesdropping on Helicanus, and subsequently lying to him. For the final scene,
from \emph{Henry VI, part 1} , the main exchange is actually between the
Countess of Auvergne and Tabolt, and so can be constituted as a false positive.

\subsection{Conclusion} 
\label{sub:conclusion}

Despite the already bleak figures generated by crunch-shake, indicating a huge
disparity between the treatment of men and women by Shakespeare, the results
could be even worse. A random sample of the scenes with females that
crunch-shake passed were all shown to be false positives on closer
examination---They all revolved around men in some way. In contrast an
examination of the scenes with males, revealed such widespread scenes as two
nobleman making fun of the newest french fashion to a general dispatching
orders to his subordinates. Only one false positive was found in the five
scenes examined.

Shakespeare is an eminent figure in English Literature, one could even say the
eminent figure. While the themes he studies in his plays are universal, it is
also clear that his plays are also singularly focused on the lives of men.
While there are notable fleshed out females in Shakespeare's play---Lady
Macbeth and Portio from the \emph{Merchant of Venice} come to mind---their
values as characters are very much defined in relation to the more important
male characters that dominate his plays. crunch-shake demonstrates not just on
an anecdotal level, but on a statistical level, just how widespread this
dominance is.

\bibliographystyle{siam} 
\bibliography{sources} 
\end{document}
